\documentclass[12pt,a4paper]{article}
%\usepackage{amsmath,amsthm,amsfonts,amssymb,mathrsfs,bm}
\usepackage{amsmath,amsthm,amsfonts,amssymb,bm,wasysym}
%\usepackage{marvosym}
\usepackage{subfigure}
\usepackage{graphicx}
\usepackage[usenames]{color}
\usepackage{verbatim}
\usepackage{epsfig}
%\usepackage{showlabels}

%\usepackage{ccfonts}
%\usepackage{euler}
%\usepackage{txfonts}
%\usepackage{mathptmx}
%\usepackage{cmbright}
%\usepackage{lucidabr}
%%\usepackage{fourier}

%\usepackage[english,greek]{babel}
%\usepackage[iso-8859-7]{inputenc}

%for soda
\topmargin 0in
\oddsidemargin .01in
\textwidth 6.5in
\textheight 9in
\evensidemargin 1in
\addtolength{\voffset}{-.6in}
\addtolength{\textheight}{0.22in}
\parskip \medskipamount
\parindent	0pt

%\oddsidemargin	0.635cm
%\textwidth	15.3cm
%\topmargin	-1cm
%\textheight	23cm
%\parindent	0pt
%\parskip 	\bigskipamount

%%%%%%%%%%%THEOREMS%%%%%%%%%%%
\newtheorem{theorem}{Theorem}[section]
\newtheorem{definition}[theorem]{Definition}
\newtheorem{assumption}[theorem]{Assumption}
\numberwithin{equation}{section}
%\newtheorem{definition}[theorem]{Definition}
%\newtheorem{defn}{Definition}[theorem]
\newtheorem{lemma}[theorem]{Lemma}
\newtheorem{proposition}[theorem]{Proposition}
%\newtheorem{corollary}{Corollary}[theorem]
\newtheorem{corollary}[theorem]{Corollary}
\newtheorem{property}{Property}[theorem]
%\newtheorem{remark}[theorem]{Remark}
\newtheorem{remark}[theorem]{Remark}
\newtheorem{example}[theorem]{Example}
\newtheorem{claim}[theorem]{Claim}

\newtheorem{question}[theorem]{Question}
\newtheorem{conjecture}[theorem]{Conjecture}
\numberwithin{equation}{section}


%%%%%%%%%%%LETTERS%%%%%%%%%%%
\def\N{\mathbb{N}}
\def\Z{\mathbb{Z}}
\def\Q{\mathbb{Q}}
\def\R{\mathbb{R}}
\def\C{\mathbb{C}}
\def\bP{\mathbb{P}}
\def\S{\mathbb{S}}
\def\AA{\mathcal{A}}
\def\F{\mathcal{F}}
\def\EE{\mathcal{E}}
\def\bE{\operatorname*{\mathbb{E}}}
\def\B{\mathcal{B}}
\def\LL{\mathcal{L}}
\def\NN{\mathcal{N}}
\def\PP{\mathcal{P}}
\def\T{\mathcal{T}}
\def\tr{{\rm{tr}}}
\renewcommand{\phi}{\varphi}
\renewcommand{\epsilon}{\varepsilon}
\def\cprime{$'$}

%\allowdisplaybreaks

%%%%%%%%%%%SYMBOLS & OPERATORS%%%%%%%%%%%
\newcommand{\1}{{\text{\Large $\mathfrak 1$}}}

\newcommand{\comp}{\raisebox{0.1ex}{\scriptsize $\circ$}}
%\newcommand{\cadlag}{{c\`adl\`ag} }
\newcommand{\bin}{\operatorname{Bin}}
%\newcommand{\var}{\operatorname{var}}
\newcommand{\cov}{\operatorname{Cov}}
%\renewcommand{\limsup}{\varlimsup}
%\renewcommand{\liminf}{\varliminf}
\newcommand{\eqdist}{\stackrel{\text{d}}{=}}
\newcommand{\dirac}{\text{\large $\mathfrak d$}}
\newcommand{\ud}{\mathrm{d}}
\newcommand{\vol}{\mathrm{vol}}
\renewcommand{\emptyset}{\varnothing}
\newcommand{\I}{_{\text{\sc i}}}
\newcommand{\II}{_{\text{\sc ii}}}
\newcommand{\III}{_{\text{\sc iii}}}
\newcommand{\IV}{_{\text{\sc iv}}}
\newcommand{\IVp}{_{\text{{\sc iv},p}}}
\newcommand{\IVf}{_{\text{{\sc iv},f}}}
\def\nn{\text{\tiny $N$}}
\newcommand{\ttt}{\widetilde}
\def\Prob{{\mathbf P}}
\def\C{{\mathcal C}}
\newcommand{\tdet}{T}
\newcommand{\sdet}{S}
\newcommand{\til}{\widetilde}

%\newcommand{\taver}{t_{\mathrm{ave}}}
%\newcommand{\tmix}{t_{\mathrm{mix}}}
%\newcommand{\tmixlazy}{\tau_1(\mathrm{lazy})}
%\newcommand{\tg}{t_{\mathrm{G}}}
%\newcommand{\tsep}{t_{\mathrm{sep}}}
%\newcommand{\tl}{t_{\mathrm{L}}}
%\newcommand{\thit}{t_{\mathrm{H}}}
%%\newcommand{\tstop}{t_{\text{\Stopsign}}}
%\newcommand{\tmax}{t_{\mathrm{hit}}}
%\newcommand{\tstop}{t_{\mathrm{stop}}}
%\newcommand{\tct}{t_{\mathrm{cts}}}
%\newcommand{\tprod}{t_{\mathrm{prod}}}
%\newcommand{\tces}{t_{\mathrm{Ces}}}
%\newcommand{\tmov}{t_{\mathrm{HIT}}}
%\newcommand{\trel}{t_{\mathrm{rel}}}

\newcommand{\pr}[1]{\mathbb{P}\!\left(#1\right)}
\newcommand{\E}[1]{\mathbb{E}\!\left[#1\right]}
\newcommand{\estart}[2]{\mathbb{E}_{#2}\!\left[#1\right]}
\newcommand{\prstart}[2]{\mathbb{P}_{#2}\!\left(#1\right)}
\newcommand{\prcond}[3]{\mathbb{P}_{#3}\!\left(#1\;\middle\vert\;#2\right)}
\newcommand{\econd}[2]{\mathbb{E}\!\left[#1\;\middle\vert\;#2\right]}
\newcommand{\econds}[3]{\mathbb{E}_{#3}\!\left[#1\;\middle\vert\;#2\right]}

\newcommand{\var}[1]{\operatorname{Var}\!\left(#1\right)}
\newcommand{\vars}[2]{\operatorname{Var}_{#2}\!\left(#1\right)}


\newcommand{\norm}[1]{\left\| #1 \right\|}
\newcommand{\fract}[2]{{\textstyle\frac{#1}{#2}}}
\newcommand{\tn}{|\kern-.1em|\kern-0.1em|}

\newcommand{\enk}[1]{\estart{#1}{n_k}}
\newcommand{\vnk}[1]{\operatorname{Var}_{n_k}\!\left(#1\right)}
\newcommand{\red}[1]{{\color{red}{#1}}}
\newcommand{\2}[1]{{\text{\Large $\mathfrak 1$}\!\left(#1\right)}}


% Martin's macros
\newcommand\be{\begin{equation}}
\newcommand\ee{\end{equation}}
\def\bZ{\mathbb{Z}}
\def\Reff{R_{\rm{eff}}}
\def\bP{\mathbb{P}}
\def\Ca{ {\mathop{\rm Comb}}(\Z, \alpha)}
\def\eps{\varepsilon}
\def\p{\Psi}



\usepackage{float}
\author{
{\sf Christian Borgs}\thanks{Microsoft Research New England e-mail: {\tt Christian.Borgs@microsoft.com}. }
\and
{\sf Jennifer Chayes}\thanks{Microsoft Research New England e-mail: {\tt jchayes@microsoft.com}.}
\and
{\sf Adam Smith}\thanks{Boston University; e-mail: {\tt ads22@bu.edu}. }
\and
{\sf Ilias Zadik}\thanks{MIT; e-mail: {\tt izadik@mit.edu}. Research done in part while an intern at Microsoft Research New England. }
}

\begin{document}



\title{On Privately Estimating Graphs}
\date{}

\maketitle

\begin{abstract}

\end{abstract}


\section{Introduction}


\section{Main Results}

\subsection{Private Graphon Estimation}
%In \cite{Borgs2015} the following theorem was established. 
%
%\begin{theorem}[Theorem 5 from \cite{Borgs2015}] 
%Let $W: [0,1]^2 \rightarrow [0,\Lambda]$ be a normalised graphon, $G=G_n(\rho W)$, $ \epsilon>0$, $\lambda \geq 1$ and $k \in \mathbb{Z}$. Assume also that
% \begin{itemize}
%\item  $6 \log n/n<\rho \leq 1/\Lambda$
%\item $8 \Lambda \leq \lambda  \leq \sqrt{n}$
%\item $2 \leq k \leq \min \{n,\sqrt{\frac{\rho}{2}},e^{\frac{\rho n}{2}}\}$
%\item $\rho n \epsilon \rightarrow + \infty$
%\end{itemize} 
%Then the Algorithm 1 from \cite{Borgs2015} with input $\epsilon,\lambda,k$ and $G$ outputs a pair $(\hat{\rho},\hat{B}) \in \mathbb{R} \times \mathbb{R}^{k \times k}$ with
%\begin{equation}
%\hat{\delta}_2\left(\frac{1}{\hat{\rho}}\hat{B},H_n(W)\right) \leq \hat{\epsilon}_k^{(O)}(H_n(W))+O_P\left(\sqrt[4]{\lambda^2(\frac{ \log k}{\rho n}+\frac{k^2}{\rho n^2})}+\lambda \sqrt{\frac{k^2 \log n}{n \epsilon}}+\frac{\sqrt{\lambda}}{n \rho \epsilon} \right).
%\end{equation}
%\end{theorem}

\begin{theorem}\label{stronger}
Let $W: [0,1]^2 \rightarrow [0,\Lambda]$ be a normalised graphon, $G=G_n(\rho W)$, $ \epsilon>0$, $\lambda \geq 1$ and $k \in \mathbb{Z}$. Assume also that
 \begin{itemize}
\item  $6 \log n/n<\rho \leq 1/\Lambda$
\item $8 \Lambda \leq \lambda  \leq \sqrt{n}$
\item $\rho n \epsilon \rightarrow +\infty$
\item $\epsilon=O(k^2 \log n)$
\end{itemize} 
Then the Algorithm 1 from \cite{Borgs2015} with input $\epsilon,\lambda,k$ and $G$ outputs a pair $(\hat{\rho},\hat{B}) \in \mathbb{R} \times \mathbb{R}^{k \times k}$ with
\begin{equation*}
\hat{\delta}_2\left(\frac{1}{\hat{\rho}}\hat{B},H_n(W)\right) \leq O_P\left(\hat{\epsilon}_k^{(O)}( H_n(W))+\sqrt{\lambda  \left(\frac{\log k}{ \rho n}+\frac{k^2}{\rho n^2}\right)} + \lambda  \sqrt{\frac{k^2\log n}{n \epsilon}}+ \frac{\sqrt{\lambda}}{n \rho \epsilon}\right)
\end{equation*}

\end{theorem}


In the next subsections we discuss the part of the rate which is dependent on $\epsilon$.
\section{A New Algorithm for Density Estimation in Random Graphs}
\label{density}


The rate we want to find is for $\rho \in [0,1]$, $$R_{\rho}=\min_{ \mu :(\mathcal{G}_n,\delta_v) \rightarrow (\mathcal{M},D_{\infty}) \\ \epsilon-Lipschitz} \max_{ p \in [0,\rho]} \mathbb{E}_{G \sim G_{n,p},\hat{p} \sim \mu_G}[|\hat{p}-p|]$$

\begin{proposition}\label{prop:32-fixed-m} Let $\epsilon, \rho$ be
  functions of $n$ such that $\epsilon n \rightarrow +\infty$. There
  is an $\epsilon$-node-DP algorithm $A$ such that, for all $m < \rho
\binom n 2$,

$$\bE_{G\sim G(n,m)} \Big| A(G) - m \Big| =  O \left(\max\left\{\rho,\sqrt{\tfrac{\log n}{n}}\right\}  \cdot\frac{\sqrt{\log n}}{n^{\frac{3}{2}} \epsilon} \right).$$
\end{proposition}



\begin{theorem}\label{thm32}
Suppose $\epsilon n \rightarrow +\infty$. For every $\rho \in
[0,\rho]$ we have $$R_{\rho} \leq O \left(
  \frac{1}{n}+\max\{\rho,\sqrt{\frac{\log n}{n}}\} \frac{\sqrt{\log n}}{n^{\frac{3}{2}} \epsilon} \right).$$
\end{theorem}


%%%%%%%%%%%%%%%%%%%%%%%%%%%%%%%%%%%%%%%%
%
% This code imports the lower bounds section
%

\section{Lower bounds for desinty estimation in one- and two- block graphs}
\label{sec:lower}

\begin{remark} Our bounds apply not only to the particular measure of
  distance on distributions used in defining node-differential
  privacy, but to any measure of distance that implies that two
  distributions cannot be reliably distinguished. A randomized
  algorithm $A$ is $\epsilon$-node-TV-stable if for all graphs $G$ and
  $G'$ that differ by the rewiring of one vertex, the distributions of
  $A(G)$ and $A(G')$ are $\tfrac 1 3$-close in \emph{total variation
    distance} (or statistical difference). 
\end{remark}

\subsection{Lower bounds for $G(n,m)$} \label{sec:lower-one-block}

\begin{proposition}
  Let $n$ be sufficiently large, and $k = \tfrac 1 2 \sqrt{n \log n}$. Let $m = {\binom n 2} - \frac k 2$
  Let $P = G(n,m)$ and $Q = G(n,m +k )$.   
  There exists a coupling of $(G,H)$ of $P$ and $Q$ such that, with probability at least $9/10$, one can obtain $H$ from $G$ by rewiring one vertex.  
\end{proposition}

\begin{proof}
  To be written.
\end{proof}

\begin{theorem}
  There exists a constant $\epsilon_0>0$ such that no
  $\epsilon_0$-node DP private algorithm can distinguish $G(n,m)$ from
  $G(n,m +k )$, for $k = \Theta(\sqrt{n \log n})$ and $m = \binom n 2
  - \tfrac k 2$. In particular, the upper bound of
  Proposition~\ref{prop:32-fixed-m} is tight for constant $\epsilon$
  and $\rho$. 
\end{theorem}

% \begin{proof}
%   First, if $G(n,m)$ and
%   $G(n,m +k )$ are not reliably distinguishable by node-private
%   algorithms, then no node private algorithm can estimate $m$ to
%   within error $k/2$ with high probability. 
% \end{proof}

\subsection{Lower bounds for general graphons}
\label{sec:lower-one-block}

We show that one cannot achieve the $\max(\frac 1 n, \frac 1
{\epsilon n^{3/2}}$ rate in general graphons. Even for the simplest
non-constant graphons, those with 2 blocks of similar size,
node-private algorithms incur error $\frac1 {\epsilon n}$ when
estimating the density. 

\begin{theorem}
  There exists a distribution $P$ on bounded 2-block graphons such that 
  for all $\eps$-DP algorithms $A$, the error in estimating the
  density of $W$ (equivalently, the expected edge density of graphs
  drawn from $G_n(W)$) is at least $\Omega(\max (\frac{1}{\sqrt{n}},
  \frac 1 {\epsilon n}))$.
\end{theorem}

\begin{proof}
  Given a parameter $q\in [0,1]$, let $W_q:[0,1]^2\to [0,1]$ be the
  graphon given by 
  $$W_q(x,y) =
  \begin{cases}
    1 & \text{if } x, y \leq q \text{ or } x,y \geq q\, , \\
    0 & \text{otherwise.}
  \end{cases}$$
  This is a 2-block graphon with blocks of sizes $q$ and $1-q$,
  respectively. The graphs generated from $W_q$ consist of two
  cliques (of size roughly $qn$ and $(1-q)n$) so it is easy to know
  which vertices belong to the same block. 

  The density of $W_q$ is
  $q^2+ (1-q)^2 = \frac 1 2 + 2 (q-\frac 1 2)^2$. Consider an
  algorithm that, given $G\sim G_n(W)$, aims to estimate the density
  of $W_q$. We can use its output $\hat \rho$ to estimate $q$ by
  setting $\hat q = \frac 1 2 - \sqrt{\frac{\hat \rho - \frac 1
      2}{2}}$. This function's derivative is finite and nonzero as
  long as $\hat \rho$ is bounded away from $\frac 1 2$. Thus, an
  algorithm that can estimate $\rho$ within error $\alpha$ on samples
  from $W_q$ can be used to estimate $q$ up to error $O(\alpha)$ (as
  long as $\hat \rho$ is bounded away from 1/2). 

  

  

\end{proof}

% \begin{claim}
%   For all real numbers $x$, we have
%   $$ (x+1) (x-1) = x^2 -1.$$
% \end{claim}

% \begin{proof}
%   \begin{eqnarray*}
%   (x+1) (x-1) 
%     & = & (x+1) x - (x+1) \\
%     & = & x^2 + x - x - 1 \\
%     & = & x^2 - 1 \\
%   \end{eqnarray*}
% \end{proof}

% For example, when $x=5$, the claim tells us that $6\times 4 = 5^2 -1$,
% which is $24$.








%%% Local Variables:
%%% mode: latex
%%% TeX-master: "FOCS"
%%% End:


%%%%%%%%%%%%%%%%%%%%%%%%%%%%%%%%%%%%%%%%



\section{A General Extension Technique}
\subsection*{The Model}
Let $n \in \mathbb{N}$ and $\epsilon>0$. We assume that the analyst's objective is to estimate a certain quantity which takes values in $ \mathbb{R}^n$ from input data which take values in a metric space $(\mathcal{M},d)$. The analyst is assumed to use for this task a randomized algorithm $\mathcal{A}$ which should be \begin{itemize}
\item[(1)] as highly \textbf{accurate} as possible for input data belonging in some \textit{hypothesis set} $\mathcal{H} \subseteq \mathcal{M}$;

\item[(2)] $\epsilon$-\textbf{differentially private} on the whole metric space of input data $(M,d)$.
\end{itemize}

In this section we state the following result. Consider an arbitrary $\epsilon$-differentially private algorithm defined on input belonging in some set $\mathcal{H} \subset \mathcal{M}.$ We show that it \textbf{can be always extended} to a $2\epsilon$-differentially private algorithm defined for arbitrary input data from $\mathcal{M}$ with the property that if the input data belongs in $\mathcal{H}$ the distribution of output values is the same with the original algorithm. We state formally the result. 

\begin{proposition}\label{extension}
Let $\hat{\mathcal{A}}$ be a randomized algorithm designed for input from $\mathcal{H} \subseteq \mathcal{M}$. Then there exists a randomized algorithm $\mathcal{A}$ defined on the whole input space $\mathcal{M}$ which is $2\epsilon$-differentially private and satisfies that for every $D \in \mathcal{H}$, $\mathcal{A}(D) \overset{d}{=}  \hat{\mathcal{A}}(D)$.
\end{proposition}

\section{Proofs}

\section{Discussion and Open Problems}


%\section{Proof of Graphon Estimation}
%The essential improvement is coming from improving Proposition 1 of \cite{Borgs2015} to the following proposition.
%\begin{proposition}\label{prop}
%Let $r \in [0,1]$, $Q \in [0,r]^{k \times k}$ be a symmetric matrix with vanishing diagonal and $A \sim \mathrm{Bern_0}(Q)$. If $\hat{B} \in \mathcal{B}_r$ satisfies 
%\begin{equation}\label{eq:condition}
%\mathrm{Score}\left(\hat{B},A\right) \geq \max_{B \in \mathcal{B}_r}[\mathrm{Score}\left(\hat{B},A\right)]-\nu^2
%\end{equation}
%for some $\nu>0$ then with probability at least $1-\exp\left(-\Omega(k \log n)\right)$
%\begin{equation*}
%\hat{\delta}_2\left(\hat{B},Q\right) \leq O\left(\epsilon_k^{(O)}(Q)+\nu+\sqrt{r \left(\frac{\log k}{n}+\frac{k^2}{n^2}\right)}  \right)
%\end{equation*}
%and in particular,
%\begin{equation*}
%\|\hat{B}\|_2 \leq O\left(\|Q\|_2+\nu+\sqrt{r \left(\frac{\log k}{n}+\frac{k^2}{n^2}\right)}  \right)
%\end{equation*}
%\end{proposition}
%
%
%
%\begin{proof} (of Proposition \ref{prop})
%Let $B_0 \in \mathcal{B}_r $ and  $\pi_0$ $k$-equipartition of $[n]$ such that 
%
%\begin{equation}\label{eq:tsyba1}
%\|(B_0)_{\pi_0}-A\|_2=\min_{B \in \mathcal{B}_r,\pi}\|B_{\pi} -A\|_2^2
%\end{equation} 
%From \cite{TsybK} we know that with probability at least $1-\exp\left(-\Omega(k \log n)\right)$
%\begin{equation}\label{eq:tsybakov}
%\|(B_0)_{\pi_0}-Q\|_2^2 =O\left(\hat{\epsilon}_k^{(O)}(Q)^2+r\left(\frac{\log k}{n}+\frac{k^2}{n^2}\right) \right)
%\end{equation}
%We first notice that equation (\ref{eq:condition}) implies that for some $\hat{\pi}$, $k$-equipartition of $[n]$ \begin{equation}\label{eq:us}
%\|\hat{B}_{\hat{\pi}}-A\|_2^2 \leq \min_{B \in \mathcal{B}_r,\pi}\|B_{\pi} -A\|_2^2 +\nu^2.
%\end{equation}
%
%
%Following an \textbf{identical} path as in the proof of \cite{TsybK} where from inequality (\ref{eq:tsyba1}) they prove inequality (\ref{eq:tsybakov}), it is easy to see that using (\ref{eq:condition}) we get again almost (\ref{eq:tsybakov}) with the only difference the addition of the parameter $\nu^2$ in the left hand side, that is with probability at least $1-\exp\left(-\Omega(k \log n)\right)$ \begin{equation}\label{eq:target}
%\|(\hat{B})_{\hat{\pi}}-Q\|_2^2 =O\left(\hat{\epsilon}_k^{(O)}(Q)^2+r\left(\frac{\log k}{n}+\frac{k^2}{n^2}\right)+\nu^2 \right).
%\end{equation} This last inequality implies easily both of the desired results.
%
%For completeness we describe in more detail how we get the corresponding result following the exact same proof as in \cite{TsybK}. In \cite{TsybK} they establish 
%\begin{itemize}
%\item[(1)] With probability at least $1-\exp\left(-\Omega(k \log n)\right)$ for all $\pi$ $k$-partition of $[n]$,
%\begin{equation}\label{eq:24}
%<Q_{\pi}-Q,Q-A> \leq \frac{1}{16}\|Q_{\pi}-Q\|_2^2+O_P \left( r \frac{\log k}{n}\right)
%\end{equation}
%
%This follows from the probabilistic bound above of equation (25), page 10 in \cite{TsybK}.
%\item[(2)] With probability at least $1-\exp\left(-\Omega(k \log n)\right)$ for all $C \in \mathcal{B}_{r}$ and $\pi $ $k$-partition of $[n]$,
%\begin{equation}\label{eq:25}
%<C_{\pi},Q-A> = \frac{1}{16}\|C\|_2^2+O_P \left( r( \frac{\log k}{n}+\frac{k^2}{n^2}  )\right) 
%\end{equation}
%
%The proof is identical to the second part of the Proof of Proposition 2.3. (second paragraph and below) in \cite{TsybK} for $C_{\pi}$ equal to what they denote as $\hat{\Theta}^r-\tilde{\Theta}_{\hat{z}_r}$.
%
%
%%As in [Tsybakov] we first notice that f 
%%q$$<C_{\pi},Q-A> \lew  
%%Indeed we know that with probability at least (?) for all $\pi$ and for all $V \in C^*_\pi$ defined at the bottom of page 12 in [Tsybakov] where $\pi$ is denote by $\hat{z}_r$ there, the inequality
%%\begin{equation}\label{eq:32}
%%<V,Q-A> \leq \frac{1}{50}\|V\|_2^2+O_P \left( r( \frac{\log k}{n}+\frac{k^2}{n^2}  )\right).
%%\end{equation}
%% Now the same proof as for Lemma 4.1. in [Tsybakov] establishes the following result
%%\begin{lemma}
%%If $\|C_{\pi}\|_2 \geq \frac{2r}{n^2}$ then there exists a $D_{\pi} \in C^*_{\pi}$ with $\|C_{\pi}-D_{\pi}\|_2 \leq \frac{1}{4}\|C_{\pi}\|_2^$ and $\|C_{\pi}-D_{\pi}\|_{\infty} \leq r$.
%%\end{lemma}
%
%\end{itemize}
%
%
%We now choose $Q_{\pi_1}$ the best (wrt to $L_2$ norm) $k$-block approximation for $Q$ so that $\|Q_{\pi_1}-Q\|_2=\hat{\epsilon}_k^{(O)}(Q)$. Now (\ref{eq:us}) since $Q_{\pi_1} \in \mathcal{B}_r$ gives
%\begin{align*}
%\|\hat{B}_{\hat{\pi}}-A\|_2^2 \leq \|Q_{\pi_1}-A\|_2^2+\nu^2
%\end{align*} 
%which now adding and substracting $\hat{Q}$ inside the norm and expanding the 2-norms gives
%\begin{align*}
%\|\hat{B}_{\hat{\pi}}-Q\|_2^2 \leq \|Q_{\pi_1}-Q\|_2^2+\nu^2+2<Q_{\pi_1}-\hat{B}_{\hat{\pi}},Q-A>
%\end{align*} 
%or
%\begin{align*}
%\|\hat{B}_{\hat{\pi}}-Q\|_2^2 \leq \|Q_{\pi_1}-Q\|_2^2+\nu^2+2<Q_{\pi_1}-Q,Q-A>+2<Q-Q_{\hat{\pi}},Q-A>+2<(Q-\hat{B})_{\hat{\pi}},Q-A>.
%\end{align*} 
%Bounding now the first two inner products according to (\ref{eq:24}) and the last according to (\ref{eq:25}) we get with probability at least $1-\exp\left(-\Omega(k \log n)\right)$ the quantity $\|\hat{B}_{\hat{\pi}}-Q\|_2^2$ is at most
%
%\begin{align*}
%\|Q_{\pi_1}-Q\|_2^2+\nu^2+ \frac{2}{16}\left(\|Q_{\pi_1}-Q\|_2^2+\|Q_{\hat{\pi}}-Q\|_2^2+\|(Q-\hat{B})_{\hat{\pi}}\|_2^2 \right) + O_P \left( r( \frac{\log k}{n}+\frac{k^2}{n^2}  )\right)  
%\end{align*}
%or recalling the way $Q_{\pi_1}$ was chosen, $\|Q_{\pi_1}-Q\|_2=\hat{\epsilon}_k^{(O)}(Q)$, we conclude
%\begin{align*}
%\|\hat{B}_{\hat{\pi}}-Q\|_2^2 \leq \frac{1}{8}\left(\|Q_{\hat{\pi}}-Q\|_2^2+\|(Q-\hat{B})_{\hat{\pi}}\|_2^2 \right) + O_P \left( \hat{\epsilon}_k^{(O)}(Q)^2+ r( \frac{\log k}{n}+\frac{k^2}{n^2}  )+\nu^2\right) . 
%\end{align*}
%
%Now recall that $Q_{\pi}$ is the best $\pi$-block approximation in $L_2$ of $Q$. This implies $$\|Q_{\pi}-Q\|_2 \leq \|\hat{B}_{\hat{\pi}}-Q\|_2$$ and by triangle inequality  also $$\|(Q-\hat{B})_{\hat{\pi}}\|_2 \leq\|\hat{B}_{\hat{\pi}}-Q\|_2+\|Q_{\pi}-Q\|_2 \leq  2\|\hat{B}_{\hat{\pi}}-Q\|_2.$$ Hence putting the last inequalities together we have proven with probability at least (?), 
%\begin{align*}
%\|\hat{B}_{\hat{\pi}}-Q\|_2^2 \leq \frac{5}{8}\|\hat{B}_{\hat{\pi}}-Q\|_2^2 + O_P \left( \hat{\epsilon}_k^{(O)}(Q)^2+ r( \frac{\log k}{n}+\frac{k^2}{n^2}) + \nu^2 \right)  
%\end{align*}
%or 
%\begin{equation*}
%\|(\hat{B})_{\hat{\pi}}-Q\|_2^2 =O\left(\hat{\epsilon}_k^{(O)}(Q)^2+r\left(\frac{\log k}{n}+\frac{k^2}{n^2}\right)+\nu^2 \right),
%\end{equation*}
%as we wanted.
%\end{proof}
%
%Here we present a proof for how Proposition \ref{prop} implies Theorem \ref{stronger}.
%\begin{proof} (of Theorem \ref{stronger})
%As in the beginning of the Proof of Theorem 5 in \cite{Borgs2015} with probability at least $1-O(\frac{\Lambda}{n})-e^{-\Omega(n \rho \epsilon)}$,  $\hat{B}$ satisfies (\ref{eq:condition}) for $Q=\rho H_n( W)$, $r=\lambda \rho$ and $\nu=O\left(\lambda \rho  \sqrt{\frac{k^2\log n}{n \epsilon}}\right)$.
%Therefore using Proposition \ref{prop} with probability at least $1-\exp(-\Omega( k\log n))-O(\frac{\Lambda}{n})-e^{-\Omega(n \rho \epsilon)}$ we know
%\begin{align*}
%&\hat{\delta}_2\left(\hat{B},\rho H_n(W)\right) \leq O\left(\epsilon_k^{(O)}(\rho H_n(W))+\sqrt{\lambda \rho \left(\frac{\log k}{n}+\frac{k^2}{n^2}\right)} + \lambda \rho  \sqrt{\frac{k^2\log n}{n \epsilon}}\right)\\
%\text{ or }&  \hat{\delta}_2\left(\frac{1}{\rho}\hat{B}, H_n(W)\right) \leq O\left(\hat{\epsilon}_k^{(O)}(H_n(W))+\sqrt{\lambda  \left(\frac{\log k}{\rho n}+\frac{k^2}{\rho n^2}\right)} + \lambda  \sqrt{\frac{k^2\log n}{n \epsilon}}\right).
%\end{align*}
%Using triangle inequality this easily gives
%\begin{align*}
% \hat{\delta}_2\left(\frac{1}{\hat{\rho}}\hat{B}, H_n(W)\right) \leq O\left(\hat{\epsilon}_k^{(O)}(H_n(W))+\sqrt{\lambda  \left(\frac{\log k}{\rho n}+\frac{k^2}{\rho n^2}\right)} + \lambda  \sqrt{\frac{k^2\log n}{n \epsilon}}+\|B\|_2 |\frac{\rho}{\hat{\rho}}-1|\right)\\
%\end{align*}
%Now Proposition \ref{prop} gives that with probability at least $1-e^{-\Omega(n \rho \epsilon)}$ it holds  $$\|B\|_2  \leq O( \|H_n(W)\|_2+\lambda \rho \sqrt{\frac{k^2\log n}{n \epsilon}}+\sqrt{\lambda \rho \left(\frac{\log k}{n}+\frac{k^2}{n^2}\right)}  ).$$
%Because $\|W\|_1=1,\|W\|_{\infty} \leq \lambda$ we have $\|H_n(W)\|_2 \leq O_P( \sqrt{\lambda})$.  Therefore $\hat{\delta}_2\left(\frac{1}{\hat{\rho}}\hat{B},H_n(W)\right)$ is at most
%
%\begin{equation*}
%O\left(\hat{\epsilon}_k^{(O)}( H_n(W))+\sqrt{\lambda  \left(\frac{\log k}{ \rho n}+\frac{k^2}{\rho n^2}\right)} + \lambda  \sqrt{\frac{k^2\log n}{n \epsilon}}+(\sqrt{\lambda}+\lambda \rho \sqrt{\frac{k^2\log n}{n \epsilon}}+\sqrt{\lambda \rho \left(\frac{\log k}{n}+\frac{k^2}{n^2}\right)} ) | \frac{\rho}{\hat{\rho}}-1|\right)
%\end{equation*}
%
%Since with probability $1-e^{-\Omega(n \rho \epsilon)} $ Lemma 8 of \cite{Borgs2015} holds we have with the same probability $| \frac{\rho}{\hat{\rho}}-1|=O( \min\{\frac{1}{n \rho \epsilon}+\sqrt{\frac{\lambda}{n }},\frac{1}{\rho}\})$. Hence with probability $1-e^{-\Omega(n \rho \epsilon)} $ using also our upper bound on $\epsilon$ we have,
%$$\sqrt{\lambda}| \frac{\rho}{\hat{\rho}}-1| \leq O(\frac{\sqrt{\lambda}}{n \rho \epsilon}+\frac{\lambda}{\sqrt{n} })=O(\frac{\sqrt{\lambda}}{n \rho \epsilon}+\lambda  \sqrt{\frac{k^2\log n}{n \epsilon}})$$ and
%$$(\lambda \rho \sqrt{\frac{k^2\log n}{n \epsilon}}+\sqrt{\lambda \rho \left(\frac{\log k}{n}+\frac{k^2}{n^2}\right)} ) | \frac{\rho}{\hat{\rho}}-1| \leq O(\sqrt{\lambda  \left(\frac{\log k}{\rho n}+\frac{k^2}{\rho n^2}\right)} + \lambda  \sqrt{\frac{k^2\log n}{n \epsilon}})$$ Combining the last three inequalities together we get that with probability at least $1-\exp(-\Omega( k\log n))-O(\frac{\Lambda}{n})-O(e^{-\Omega(n \rho \epsilon)}),$
%\begin{equation*}
%\hat{\delta}_2\left(\frac{1}{\hat{\rho}}\hat{B},H_n(W)\right) \leq O\left(\hat{\epsilon}_k^{(O)}( H_n(W))+\sqrt{\lambda  \left(\frac{\log k}{ \rho n}+\frac{k^2}{\rho n^2}\right)} + \lambda  \sqrt{\frac{k^2\log n}{n \epsilon}}+ \frac{\sqrt{\lambda}}{n \rho \epsilon}\right),
%\end{equation*}as we wanted.
%
%
%\end{proof}
%
%\section{Proof of Upper Bound}
%We start with a lemma.
%
%\begin{lemma}\label{union}
%Let $p \in [0,1]$. For every $S \subseteq V(G),|S|=k$ set the event $$A_{p,S}:=\{ |E(S,S^c)+E(S,S)-p\left[ k\left(n-k\right)+\binom{k}{2} \right]| \leq \max\{p(1-p),\sqrt{\frac{\log n}{n}}\}2k\sqrt{n \log n} |\}.$$ Then it holds
%$$\mathbb{P}_{G \sim G_{n,p}}\left[ \bigcup_{S \subseteq V(G) } A^c_{p,S} \right] \leq \frac{1}{n^2}.$$
%\end{lemma}
%
%\begin{proof}
%
%Set $c=\max\{p(1-p),\sqrt{\frac{\log n}{n}}\}$.
%By a union bound, Berstein inequality and basic algebra we have
%\begin{align*}
%&\mathbb{P}_{G \sim G_{n,p}}\left[ \bigcup_{S \subseteq V(G) } A^c_{p,S} \right] \\
%&\leq \sum_{k=1}^{n} \binom{n}{k} \exp\left( -4\frac{c^2k^2n \log n}{\left(k(n-k)+\binom{k}{2}\right)p(1-p)+2c k \sqrt{n \log n}} \right)\\
%&\leq \sum_{k=1}^n n^k n^{-4 k} \\
%&\leq n\frac{1}{n^3}=\frac{1}{n^2}
%\end{align*}
%
%\end{proof}
%
%\begin{proof}
%We now begin the proof of Proposition (\ref{thm32}). We remind the reader that the sampling error is $\frac{1}{n}$.
%
%
%Given the Proposition (\ref{extention}) a strategy would be to find a subset $H_n$ of all the graphs on $n$ vertices so that
%\begin{equation}\label{eq:assum1}
%\max_{p \in [0,1]} \mathbb{P}_{G \sim G_{n,p}} \left( G \not \in H_n \right) \leq \frac{1}{n}
%\end{equation}
%
%and furthermore define an $\epsilon$-Lip function $\hat{\mu}$ on $H_n$ so that for all $G \in H_n$, 
%\begin{equation}\label{eq:assum2}
%\mathbb{E}_{\hat{p} \sim \hat{\mu}_G}[|\hat{p}-e(G)|] \leq \max\{p(1-p),\sqrt{\frac{\log n}{n}}\} \frac{100}{n^{\frac{3}{2}} \epsilon} \sqrt{\log n}
%\end{equation}
%
%Then, from Proposition (\ref{extention}) we could extend this mapping to a $2\epsilon$-Lipschitz mapping on the space of all graphs and furthermore have for all $p$,
%\begin{align*}
%&\mathbb{E}_{G \sim G_{n,p},\hat{p} \sim \mu_G}[|\hat{p}-e(G)|]\leq  \mathbb{P}_{G \sim G_{n,p}} \left( G \not \in H_n \right)+\max_{G \in H_n} \mathbb{E}_{\hat{p} \sim \hat{\mu}_G}[|\hat{p}-e(G)|]\\
%&=O \left(\frac{1}{n}+\max\{p(1-p),\sqrt{\frac{\log n}{n}}\} \frac{\sqrt{\log n}}{n^{\frac{3}{2}}\epsilon} \right)
%\end{align*}
%
%Given lemma (\ref{union}) we define $$H_n=\bigcup_{p \in [0,1]}\bigcap_{S \subseteq V(G)} A_{p,S},$$ that is all the graphs on $n$ vertices for which for some $p \in [0,1]$  all $A_{p,S}$ are satisfied. This represents for us the class of \textbf{homogeneous} graphs. Given the Lemma (\ref{union}) we know that indeed (\ref{eq:assum1}) is satisfied.
%
%For the next condition we define for every graph $G \in H_n$ the distribution over $[0,1]$ to come from the addition of ``truncated" Laplacian noise given by $$\hat{\mu}_G(q) \propto 2^{-\epsilon c \min \{\frac{n^{\frac{3}{2}}}{\max\{p(1-p),\sqrt{\frac{\log n}{n}}\}\sqrt{\log n}} |e(G)-q|,n\}}$$ for $q \in [0,1]$. The constant $c>0$ will be satisfied later on.
%
%It is easy to prove that (\ref{eq:assum2}) is satisfied but we need to prove that our mapping is $\epsilon$-Lip.
%
%To do this it is easy to establish first by triangle inequality that for any graphs on $n$ vertices $G_1,G_2$ $$D_{\infty} \left( \hat{\mu}_{G_1},\hat{\mu}_{G_2} \right) \leq 2\epsilon c \min \{\frac{n^{\frac{3}{2}}}{\max\{p(1-p),\sqrt{\frac{\log n}{n}}\}\sqrt{\log n}} |e(G_1)-e(G_2)|,n\} $$
%
%Hence we only need to prove that for some $c>0$ small enough and for any $G_1,G_2 \in H$ it holds $$c\min\{ \frac{n^{\frac{3}{2}}}{\max\{p(1-p),\sqrt{\frac{\log n}{n}}\}\sqrt{\log n}}|e(G)-e(G')|,n\} \leq \delta_V(G,G').$$ 
%
%This is what we prove in the next claim we completes the proof.
%
%\begin{claim}
%There exists a universal constant $c>0$ such that for any $G,G' \in H$, it holds $$c\min\{ \frac{n^{\frac{3}{2}}}{\max\{p(1-p),\sqrt{\frac{\log n}{n}}\} \sqrt{\log n}}|e(G)-e(G')|,n\} \leq \delta_V(G,G').$$ 
%\end{claim}
%
%\begin{proof}
%Let $G,G' \in H$. By assuming $c<\frac{1}{4}$ we may assume that $\delta_V(G,G') \leq \frac{n}{4}$. In that case we will prove that for some universal $c>0$, $$c\frac{n^{\frac{3}{2}}}{\sqrt{\log n}}|e(G)-e(G')| \leq  \max\{p(1-p),\sqrt{\frac{\log n}{n}}\} \delta_V(G,G').$$ 
%
%Let $p,q$ such that $G \in \bigcap_{S \subseteq V(G)} A_{p,S}$ and $G'\in \bigcap_{S \subseteq V(G)} A_{q,S}$.
%Consider $S_0 \subseteq V(G)$ the vertices that need to be rewired to change $G$ to $G'$. In particular it holds $\delta_V(G,G')=|S_0|=:k$.
%Now we have
%\begin{align*}
%& |E(G)-E(G')|\\
%&=|E_G(S_0,S_0)+E_G(S_0,S^c_0)-E_{G'}(S_0,S_0)-E_{G'}(S_0,S^c_0)|\\
%& \leq |p-q| \left( k(n-k)+\binom{k}{2} \right) +4 \max\{p(1-p),\sqrt{\frac{\log n}{n}}\} k  \sqrt{n \log n} \text{ ,using  }  G \in A_{p,S_0}, G' \in A_{q,S_0}
%\end{align*}
%
%Now observe that since $G \in A_{p,V(G)}, G' \in A_{q,V(G')}$ it holds $|E(G)-p \binom{n}{2}| \leq 2 \max\{p(1-p),\sqrt{\frac{\log n}{n}}\} n \sqrt{n \log n}$, $|E(G')-q \binom{n}{2}| \leq 2 \max\{p(1-p),\sqrt{\frac{\log n}{n}}\} n \sqrt{n \log n}$. Hence, $$|p-q| \leq \frac{1}{\binom{n}{2}}|E(G)-E(G')|+\frac{4 \max\{p(1-p),\sqrt{\frac{\log n}{n}}\} n \sqrt{n \log n}}{\binom{n}{2}}$$
%
%Plugging this into the previous inequality we have, \begin{align*}
%&|E(G)-E(G')| \leq \\
%&\left[\frac{1}{\binom{n}{2}}|E(G)-E(G')|+\frac{4 \max\{p(1-p),\sqrt{\frac{\log n}{n}}\} n \sqrt{n \log n}}{\binom{n}{2}}\right] \left( k(n-k)+\binom{k}{2} \right)\\
%& +4\max\{p(1-p),\sqrt{\frac{\log n}{n}}\} k  \sqrt{n \log n},
%\end{align*}
%
%
%
%Since  $ k(n-k)+\binom{k}{2} \leq kn$ we can equivalently write the inequality as
%$$\left( \binom{n}{2}-kn \right)|e(G)-e(G')| \leq 8 \frac{n^2}{\binom{n}{2}} \max\{p(1-p),\sqrt{\frac{\log n}{n}}\} k  \sqrt{n \log n}.$$
%
%But now as we have assumed $k \leq \frac{n}{4}$ we have $\binom{n}{2}-kn \geq \frac{n^2}{8}$ (large n) and since $\frac{n^2}{\binom{n}{2}} \leq 4$ (large n) the inequality gives 
%for some universal $c>0$
%$$c\frac{n^{\frac{3}{2}}}{\sqrt{\log n}}|e(G)-e(G')| \leq  \max\{p(1-p),\sqrt{\frac{\log n}{n}}\} k  =\max\{p(1-p),\sqrt{\frac{\log n}{n}}\} \delta_V(G,G'),$$
%as we wanted.
%
%
%\end{proof}
%\end{proof}


\bibliographystyle{apalike}
\bibliography{bibliography}

\end{document}